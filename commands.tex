% DE: wird fuer Tabellen benötigt (z.B. >{centering\RBS}p{2.5cm} erzeugt einen zentrierten 2,5cm breiten Absatz in einer Tabelle
\newcommand{\RBS}{\let\\=\tabularnewline}

% EN: To avoid issues with Springer's \mathplus
%     See also http://tex.stackexchange.com/q/212644/9075
\providecommand\mathplus{+}

% DE: typoraphisch richtige Abkürzungen
\newcommand{\zB}{z.\,B.\xspace}
\newcommand{\bzw}{bzw.\xspace}
\newcommand{\usw}{usw.\xspace}
\renewcommand{\dh}{d.\,h.\xspace}

% EN: from hmks makros.tex - \indexify
\newcommand{\toindex}[1]{\index{#1}#1}

% DE: Tipp aus "The Comprehensive LaTeX Symbol List"
\newcommand{\dotcup}{\ensuremath{\,\mathaccent\cdot\cup\,}}

% DE: Anstatt $|x|$ $\abs{x}$ verwenden.
%     Die Betragsstriche skalieren automatisch, falls "x" etwas größer sein sollte...
\newcommand{\abs}[1]{\left\lvert#1\right\rvert}

% DE: für Zitate
\newcommand{\citeS}[2]{\cite[S.~#1]{#2}}
\newcommand{\citeSf}[2]{\cite[S.~#1\,f.]{#2}}
\newcommand{\citeSff}[2]{\cite[S.~#1\,ff.]{#2}}
\newcommand{\vgl}{vgl.\ }
\newcommand{\Vgl}{Vgl.\ }

% EN: For the algorithmic package
\newcommand{\commentchar}{\ensuremath{/\mkern-4mu/}}
\algrenewcommand{\algorithmiccomment}[1]{\hfill $\commentchar$ #1}

% DE: Seitengrößen - Gegen Schusterjungen und Hurenkinder...
\newcommand{\largepage}{\enlargethispage{\baselineskip}}
\newcommand{\shortpage}{\enlargethispage{-\baselineskip}}

\newcommand{\initialism}[1]{%
  \ifdeutsch%
    \textsc{#1}\xspace%
  \else%
    \textlcc{#1}\xspace%
  \fi%
}
\newcommand{\OMG}{\initialism{OMG}}
\newcommand{\BPEL}{\initialism{BPEL}}
\newcommand{\BPMN}{\initialism{BPMN}}
\newcommand{\UML}{\initialism{UML}}
